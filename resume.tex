%!TEX program = lualatex
\documentclass[10pt, a4paper]{article}

\usepackage{fontspec}
\usepackage{polyglossia}
\usepackage[left=1.5cm, right=1.5cm, vmargin=1.5cm, noheadfoot, marginparwidth=0cm]{geometry}
\usepackage[final]{microtype}
\usepackage{titlesec}
\usepackage{color}
\usepackage{grid-system}
\usepackage{amssymb}
\usepackage[svgnames]{xcolor}
\usepackage{hyperref}
\usepackage{marvosym}

\setmainfont[Ligatures=TeX]{Calibri}
\setmonofont{Alma Mono}
\setlength{\parindent}{0pt}
\titleformat{\section}{\Large\scshape\raggedright\color{red}}{}{0em}{}
\newcommand*{\greysquare}{\textcolor{gray}{\blacksquare}}
\hypersetup{
     colorlinks   = true,
     citecolor    = gray,
     urlcolor=Blue
}

\begin{document}

%===============================================================================
% Kontakt info
%===============================================================================
\begin{minipage}[t][2cm][t]{.5\textwidth}
\begin{tabular}{@{}l@{}} % @{} removes tabular outer padding
\\
{\Huge\scshape Carsten Jørgensen} \\[2.35ex]
{\LARGE\scshape lead software engineer og arkitekt}
\end{tabular}
\end{minipage}
\hfill
\begin{minipage}[t][2cm][t]{.4\textwidth}
\hfill
\texttt{
\begin{tabular}{l}
Badstuestræde 18D, 1th \\
1209 København K \\[1ex]
\Telefon{ +45 2971 2097} \\
\Letter{ \href{mailto:carstenj@gmail.com}{carstenj@gmail.com}} \\
\Mundus{ \href{http://carstenj.io/}{http://carstenj.io/}}
\end{tabular}
}
\end{minipage}
%==============================================================================
% Erhvervserfaring
%==============================================================================
\section{Erhvervserfaring}
\begin{Row}%
  \begin{Cell}{1}
    \textbf{Lead Software Engineer} \\ [1ex]
    Danske Bank \\
    Jan 2016 -- % chktex 8
  \end{Cell}
  \begin{Cell}{3}
    Ansat i Derivative Risk and IT, hvor jeg arbejder med \\ [1ex]
    $\greysquare$ full-stack udvikling af post trade services til både internt
    brug og eksterne kunder \\
    $\greysquare$ udvikling og devops på bankens interne model til Value-at-Risk
     beregninger \\
  \end{Cell}
\end{Row}
\\[0.5cm]\begin{Row}%
  \begin{Cell}{1}
    \textbf{Løsningsarkitekt} \\ [1ex]
    Danmarks Nationalbank \\
    Nov 2014 -- Dec 2015 % chktex 8
  \end{Cell}
  \begin{Cell}{3}
    I rollen som løsningsarkitekt var det mit ansvar at \\ [1ex]
    $\greysquare$ sikre den tekniske leverance, herunder arkitekturen, for det
    kommende nationale kreditregister med særlig fokus på IT-sikkerhed og
    datamængder \\
    $\greysquare$ bidrage til, at Nationalbankens IT-systemer udvikler sig
    konsistent og i samme retning
  \end{Cell}
\end{Row}
\\[0.5cm]
\begin{Row}
  \begin{Cell}{1}
    \textbf{Softwareudvikler \\
    og arkitekt} \\ [1ex]
    Edlund A/S \\
    Dec 2008 -- Okt 2014 % chktex 8
  \end{Cell}
  \begin{Cell}{3}
    Som softwareudvikler og arkitekt var det min opgave at udvikle software
    baseret på forsikringsmatematiske principper og \\ [1ex]
    $\greysquare$ sparre med PFA's forretningsanalytikere omkring krav til
    PFA$+$ samt udvikle og implementere den overordnede arkitektur og løsning
    primært omkring Unitlink \\
    $\greysquare$ deltage i udviklingen af nyt standard rammesystem til Nordea
    Liv og Pension \\
    $\greysquare$ have den daglige kontakt til Skandia og ansvaret for udvikling
    af deres kundespecifikke ønsker
  \end{Cell}
\end{Row}
\\[0.5cm]
\begin{Row}%
  \begin{Cell}{1}
    \textbf{Software udvikler} \\ [1ex]
    Microsoft \\
    Mar 2003 -- Nov 2008 % chktex 8
  \end{Cell}
  \begin{Cell}{3}
    I mine 5 år hos Microsoft arbejdede jeg på tre større projekter, hvor
    jeg \\ [1ex]
    $\greysquare$ havde ansvaret for front-end delen af en ny ``Sales and
    Operations Planning'' feature til næste version af AX \\
    $\greysquare$ havde medansvar i AX-kerneudvikling til Microsoft Dynamics AX
    4.0 \\
    $\greysquare$ deltog i udviklingen af innovativt ERP prototype projekt
    baseret på SOA arkitektur og en rolle baseret konfigurerbar opsætning af
    forretningsprocesser.
  \end{Cell}
\end{Row}
\\[0.5cm]
\begin{Row}%
  \begin{Cell}{1}
    \textbf{Chief Financial \\
    Consultant} \\ [1ex]
    SimCorp \\
    Okt 2000 -- Feb 2003 % chktex 8
  \end{Cell}
  \begin{Cell}{3}
    I den afdeling som har ansvaret for matematisk finansielle beregninger i
    SimCorp Dimension var det min opgave at \\ [1ex]
    $\greysquare$ deltage i refaktoreringen af beregningsmotoren fra C til C$++$
    med primært ansvar for optimeringsalgoritmer \\
    $\greysquare$ være ansvarlig for afdelingen software releases og
    konfigurationsstyring
  \end{Cell}
\end{Row}
\\[0.5cm]
\begin{Row}%
  \begin{Cell}{1}
    \textbf{Senior Kvantitativ \\
    Analytiker} \\ [1ex]
    Nordea Markets \\
    Oct 1997 -- Sep 2000 % chktex 8
  \end{Cell}
  \begin{Cell}{3}
    Som analytiker var jeg med til at \\ [1ex]
    $\greysquare$ udvikle funktionalitet og Excel løsninger til Nordea Analytics
    platformen \\
    $\greysquare$ udvikle Markets interne system til realkreditmodellering \\
    $\greysquare$ implementere, teste og konvertere et ældre system til Infinity
    Financial Technology systemet med ansvar for rentederivater
  \end{Cell}
\end{Row}
\\[0.5cm]
\begin{Row}%
  \begin{Cell}{1}
    \textbf{Softwareudvikler} \\ [1ex]
    SimCorp \\
    Aug 1995 -- Sept 1997 % chktex 8
  \end{Cell}
  \begin{Cell}{3}
    Som software udvikler var det min opgave at deltage i \\ [1ex]
    $\greysquare$ den årlige release af en række Excel add-ins \\
    $\greysquare$ koordineringen af en tværafdelings udviklingsopgave af et
    modul til Ka\-pi\-tal\-dæk\-nings\-di\-rek\-ti\-vet \\
    $\greysquare$ udviklingen af matematisk finansielt bibliotek i C med
    tilhørende unit tests
  \end{Cell}
\end{Row}

%===============================================================================
% Uddannelse afsnit
%===============================================================================
\section{Uddannelse}
\begin{Row}%
  \begin{Cell}{1}
    \textbf{Softwareudvikling} \\[1ex]
    2002 -- 2006 % chktex 8
  \end{Cell}
  \begin{Cell}{3}
    IT Universitetet. Kurser i database og distribuerede systemer, algoritmer og
    datastrukturer, programmeringssprog samt avanceret objekt-orienteret
    programmering
  \end{Cell}
\end{Row}
\\[0.5cm]
\begin{Row}%
  \begin{Cell}{1}
    \textbf{Visiting Member} \\[1ex]
    1999
  \end{Cell}
  \begin{Cell}{3}
    Courant Institute of Mathematical Sciences, New York University. Mathematics
    of Finance program. Vært Marco Avellaneda. Modtager af Andersens Rejselegat
    for Matematikere.
  \end{Cell}
\end{Row}
\\[0.5cm]
\begin{Row}%
  \begin{Cell}{1}
    \textbf{Erasmus} \\[1ex]
    1992 -- 1993 % chktex 8
  \end{Cell}
  \begin{Cell}{3}
    University of Leeds, England. Anvendt statistik og ikke-kommutativ algebra
  \end{Cell}
\end{Row}
\\[0.5cm]
\begin{Row}%
  \begin{Cell}{1}
    \textbf{Cand.Scient. Matematik \\ og Statistik} \\[1ex]
    1988 -- 1995 % chktex 8
  \end{Cell}
  \begin{Cell}{3}
    Københavns Universitet. Numerisk analyse og målteori
  \end{Cell}
\end{Row}

%===============================================================================
% Diverse
%===============================================================================
\section{Kompetencer}
\begin{Row}%
  \begin{Cell}{1}
    Programmingssprog \\[1ex]
    og frameworks
  \end{Cell}
  \begin{Cell}{3}
    Produktionserfaring med: C\#, .NET, WCP, WPF, LINQ, Rx, SQL,  % chktex 26,
    C++, STL, Boost og C \\
    Derudover: kendskab til F\#, Scala, Python, Java, JavaScript, R og
    MongoDB
  \end{Cell}
\end{Row}
\\[0.5cm]
\begin{Row}%
  \begin{Cell}{1}
    Værktøjer \\[1ex]
    og platforme
  \end{Cell}
  \begin{Cell}{3}
    Visual Studio, SQL Server, Git, Github, Mercurial, TFS, Nunit, Xamarin
    Studio, ReSharper og RStudio
  \end{Cell}
\end{Row}
\\[0.5cm]
\begin{Row}%
  \begin{Cell}{1}
    Kort fortalt
  \end{Cell}
  \begin{Cell}{3}
    21+ års erfaring med professionel software udvikling af large scale systemer
    til primært til den finansielle sektor. Erfaring med matematisk modellering
    indenfor områderne finans, optimering og supply chain management. Omfattende
    forretningsviden på tre områder: pension (6 år), finans, bank og capital
    markets (10 år) og ERP systemer (5 år). Mit kendetegn er levering til tiden
    i den ønskede kvalitet mens min største styrke er at kunne fungere som
    bindeled mellem forretning og udvikling.
  \end{Cell}
\end{Row}

%===============================================================================
% Kurser
%===============================================================================
\section{Udvalgte kurser}
\begin{Row}%
  \begin{Cell}{1}
    2017
  \end{Cell}
  \begin{Cell}{3}
    Implementing Domain-Driven Design Workshop med Vaughn Vernon
  \end{Cell}
\end{Row}
\\[0.5cm]\begin{Row}%
  \begin{Cell}{1}
    2015
  \end{Cell}
  \begin{Cell}{3}
    Advanced Distributed Systems Design med Udi Dahan
  \end{Cell}
\end{Row}
\\[0.5cm]
\begin{Row}%
  \begin{Cell}{1}
    2014
  \end{Cell}
  \begin{Cell}{3}
    Grundlæggende livsforsikringsmatematik (Liv1). Københavns Universitet
  \end{Cell}
\end{Row}
\\[0.5cm]
\begin{Row}%
  \begin{Cell}{1}
    2013
  \end{Cell}
  \begin{Cell}{3}
    Principles of Reactive Programming med Martin Odersky. Coursera
  \end{Cell}
\end{Row}
\\[0.5cm]
\begin{Row}%
  \begin{Cell}{1}
    2012
  \end{Cell}
  \begin{Cell}{3}
    Essential SQL Server 2012 for Developers. DevelopMentor
  \end{Cell}
\end{Row}
\\[0.5cm]
\begin{Row}%
  \begin{Cell}{1}
    2010
  \end{Cell}
  \begin{Cell}{3}
    Certificeret SCRUM master. Trifork A/S
  \end{Cell}
\end{Row}

\section{Anbefaling og referencer}
Se venligst min \href{https://dk.linkedin.com/in/carstenjoergensen}{LinkedIn}
profil for en række anbefalinger.
\\[0.25cm]
Personlige referencer med tidligere ledere på forespørgsel.
\end{document}
